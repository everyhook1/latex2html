% przyklad pracy w jeyku polskim

\documentclass[14pt]{article}
\usepackage[OT4]{fontenc}
\usepackage[latin2]{inputenc}

\title{Tytuł pracy}
\author{Alojzy Babel}
\date{2012}

\begin{document}

\maketitle


\section{Wstep}
Jasne ale \\ ale ``quoted'' ale \# ale \$ ale \% ale \^{} ale \& ale \_ ale \{ ale \} ale \~{} ale postawienie problemu, okreslenie \emph{Glownych Celow Pracy}.

\begin{itemize}
sdfsdfsdf
\item Co? Przedmiot, problem.

\item Jak? (Metoda , króko).

\item Dlaczego? erasa problemu badawczego.

\item Po co? Implikacje, konsekwencje, walory.

\item  Co bedzie w kolejnych rozdzialach?
\end{itemize}


\section{Podstawy teoretyczne}
Opis zagadnien zwiazanych z analizowanym problemem.

\section{Projekt}
 Okreslenie wymagan, projekt rozwiazania problemu.

\section{Implementacja}

Implementacja projektu, co zostalo zaimplementowane, jak zostalo zaimplementowane, jak zostaly spelnione wymagania.

\end{document}

